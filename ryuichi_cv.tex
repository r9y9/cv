%%%%%%%%%%%%%%%%%%%%%%%%%%%%%%%%%%%%%%%%%%%%%%%%%%%%%%%%%%%%%%%%%%%%%%%%%%%%%%%
% A clean template for an academic CV
%
% Uses tabularx to create two column entries (date and job/edu/citation).
% Defines commands to make adding entries simpler.
%
%%%%%%%%%%%%%%%%%%%%%%%%%%%%%%%%%%%%%%%%%%%%%%%%%%%%%%%%%%%%%%%%%%%%%%%%%%%%%%%

\documentclass[11pt, a4paper]{article}

% Full Unicode support for non-ASCII characters
\usepackage[utf8]{inputenc}

% Identifying information
\newcommand{\Title}{Curriculum Vit\ae}
\newcommand{\FirstName}{Ryuichi}
\newcommand{\LastName}{Yamamoto}
\newcommand{\Initials}{R}
\newcommand{\MyName}{\FirstName\ \LastName}
\newcommand{\Email}{zryuichi@gmail.com}
\newcommand{\PersonalWebsite}{r9y9.github.io}
\newcommand{\ORCID}{0000-0002-0720-0967}
\newcommand{\Affiliation}{Voice Team\\LINE Corporation}
\newcommand{\Address}{
  Yotsuya Office/Yotsuya Tower 23rd FL.\\ 1-6-1 Yotsuya, Shinjuku-ku, Tokyo
}

% Template configuration
%%%%%%%%%%%%%%%%%%%%%%%%%%%%%%%%%%%%%%%%%%%%%%%%%%%%%%%%%%%%%%%%%%%%%%%%%%%%%%%

% Disable hyphenation
\usepackage[none]{hyphenat}

% Control the font size
\usepackage{anyfontsize}

% Icon fonts (requires using xelatex or luatex)
\usepackage[fixed]{fontawesome5}
\usepackage{academicons}

% Template variables for styling
\newcommand{\TablePad}{\vspace{-0.4cm}}
\newcommand{\SoftwareTitle}[1]{{\bfseries #1}}
\newcommand{\TableTitle}[1]{{\fontsize{12pt}{0}\selectfont \itshape #1}}

% For fancy and multipage tables
\usepackage{tabularx}
\usepackage{ltablex}

% Define a new environment to place all CV entries in a 2-column table.
% Left column are the dates, right column the entries.
\usepackage{environ}
\NewEnviron{EntriesTable}{
\TablePad
\begin{tabularx}{\textwidth}{@{}p{0.12\textwidth}@{\hspace{0.02\textwidth}}p{0.86\textwidth}@{}}
  \BODY
\end{tabularx}
}

% Macros to add links and mark publications
\newcommand{\DOI}[1]{doi:\href{https://doi.org/#1}{#1}}
\newcommand{\DOILink}[1]{\href{https://doi.org/#1}{doi.org/#1}}
\newcommand{\Preprint}[1]{\newline • Preprint: \faFilePdf\ \DOILink{#1}}
\newcommand{\Youtube}[1]{\newline • Recording: \faYoutube\, \href{https://www.youtube.com/watch?v=#1}{youtube.com/watch?v=#1}}
\newcommand{\GitHub}[1]{\newline • Code: \faGithub\ \href{https://github.com/#1}{#1}}
\newcommand{\Role}[1]{\newline • Role: #1}
\newcommand{\Website}[1]{\newline • Website: \href{https://#1}{#1}}
\newcommand{\Slides}[1]{\newline • Slides: \faTv\ \href{https://#1}{#1}}
\newcommand{\SlidesDOI}[1]{\newline • Slides: \faTv\ \DOILink{#1}}
\newcommand{\PosterDOI}[1]{\newline • Poster: \faImage\ \DOILink{#1}}
\newcommand{\OA}{\aiOpenAccess\enspace}
\newcommand{\Invited}{\newline • \textbf{Invited talk}}

% Macros to set the year and duration on the left column
\newcommand{\Duration}[2]{\fontsize{10pt}{0}\selectfont #1 -- #2}
\newcommand{\Year}[1]{\fontsize{10pt}{0}\selectfont #1}
\newcommand{\Ongoing}{present}
%\newcommand{\Ongoing}{$\ast$}
\newcommand{\Future}{future}
\newcommand{\Review}{in review}
\newcommand{\Accepted}{accepted}
\newcommand{\Appointment}[4]{\textbf{#1} \newline #2 \newline #3 \newline #4}

% Define command to insert month name and year as date
\usepackage{datetime}
\newdateformat{monthyear}{\monthname[\THEMONTH], \THEYEAR}

% Set the page margins
\usepackage[a4paper,margin=1.5cm,includehead,headsep=5mm]{geometry}

% To get the total page numbers (\pageref{LastPage})
\usepackage{lastpage}

% No indentation
\setlength\parindent{0cm}

% Increase the line spacing
\renewcommand{\baselinestretch}{1.1}
% and the spacing between rows in tables
\renewcommand{\arraystretch}{1.5}

% Remove space between items in itemize and enumerate
\usepackage{enumitem}
\setlist{nosep}

% Use custom colors
\usepackage[usenames,dvipsnames]{xcolor}

% Set fonts. Requires compilation with xelatex
\usepackage{fontspec}  % required to make older xelatex compile with UTF8

% Configure the font style for sections
\usepackage{sectsty}
\sectionfont{\vspace{0.5cm}\bfseries\fontsize{12pt}{0}\selectfont\uppercase}
\subsectionfont{\vspace{0.2cm}\mdseries\fontsize{12pt}{0}\selectfont\uppercase}

% Set the spacing for sections
%\usepackage{titlesec}
%\titlespacing{\section}{0pt}{0cm}{0.3cm}
%\titlespacing{\subsection}{0pt}{0.3cm}{0.3cm}

% Disable number of sections. Use this instead of "section*" so that the sections still
% appear as PDF bookmarks. Otherwise, would have to add the table of contents entries
% manually.
\makeatletter
\renewcommand{\@seccntformat}[1]{}
\makeatother

% Set fancy headers
\usepackage{fancyhdr}
\pagestyle{fancy}
\fancyhf{}
\chead{
  \fontsize{10pt}{12pt}\selectfont
  \MyName
  \hspace{0.2cm} -- \hspace{0.2cm}
  \Title
  \hspace{0.2cm} -- \hspace{0.2cm}
  \monthyear\today
}
\rhead{\fontsize{10pt}{0}\selectfont \thepage/\pageref*{LastPage}}
\renewcommand{\headrulewidth}{0pt}

% Metadata for the PDF output and control of hyperlinks
\usepackage[colorlinks=true]{hyperref}
\hypersetup{
  pdftitle={\MyName\ - \Title},
  pdfauthor={\MyName},
  linkcolor=blue,
  citecolor=blue,
  filecolor=black,
  urlcolor=MidnightBlue
}
%%%%%%%%%%%%%%%%%%%%%%%%%%%%%%%%%%%%%%%%%%%%%%%%%%%%%%%%%%%%%%%%%%%%%%%%%%%%%%%


\begin{document}

% No header for the first page
\thispagestyle{empty}

%%%%%%%%%%%%%%%%%%%%%%%%%%%%%%%%%%%%%%%%%%%%%%%%%%%%%%%%%%%%%%%%%%%%%%%%%%%%%%%
% HEADER
{\fontsize{22pt}{0}\selectfont\MyName}\\[-0.1cm]
\rule{\textwidth}{0.2pt}
\begin{minipage}[t]{0.595\textwidth}
  \Affiliation
  \\
  \Address
\end{minipage}
\begin{minipage}[t]{0.405\textwidth}
  \begin{flushright}
  Last updated: \monthyear\today
  \\
  % ORCID: \href{https://orcid.org/\ORCID}{\ORCID}
  Publications: \href{https://scholar.google.co.jp/citations?user=PpjbClsAAAAJ\&hl=en}{Google Scholar}
  \\
  E-mail: \href{mailto:\Email}{\Email}
  \\
  Website: \href{https://\PersonalWebsite}{\PersonalWebsite}
  \end{flushright}
\end{minipage}


%%%%%%%%%%%%%%%%%%%%%%%%%%%%%%%%%%%%%%%%%%%%%%%%%%%%%%%%%%%%%%%%%%%%%%%%%%%%%%%
\section{Education}

\begin{EntriesTable}
  \Duration{2022}{\Ongoing}  & Ph.D course, Graduate School of Informatics \newline
  Nagoya University, Nagoya, Japan\newline
  Superviser: Prof. Tomoki Toda
  \\
  \Duration{2011}{2013}  & \textbf{M.Eng}, Graduate School of Engineering \newline
  Nagoya Institute of Technology, Nagoya, Japan\newline
  Superviser: Prof. Tadashi Kitamura
  \\
  \Duration{2007}{2011}  & \textbf{B.Eng}, Department of Computer Science \newline
  Nagoya Institute of Technology, Nagoya, Japan\newline
  Superviser: Prof. Tadashi Kitamura
\end{EntriesTable}

%%%%%%%%%%%%%%%%%%%%%%%%%%%%%%%%%%%%%%%%%%%%%%%%%%%%%%%%%%%%%%%%%%%%%%%%%%%%%%%
\section{Professional experience}

\begin{EntriesTable}
  \Duration{2021}{\Ongoing}  &
  \Appointment{Senior Research Scientist}{Voice Team}{LINE Corporation, Tokyo, Japan}
  \\
  \Duration{2018}{2020}  &
  \Appointment{Research Engineer}{Voice Team}{LINE Corporation, Tokyo, Japan}
  \\
  \Duration{2018}{2019}  &
  \Appointment{Research Engineer}{Clova Voice}{NAVER Corporation, Seongnam, Gyeonggi-do, Korea}
  \\  \Duration{2013}{2017}  &
  \Appointment{Software Engineer}{Computer Vision Team}{teamLab Inc., Tokyo, Japan}
\end{EntriesTable}

\section{Research areas}

\begin{itemize}

    \item Statistical Speech Synthesis, Machine Learning, Voice Conversion
    \item Music Signal Processsing, Music Information Retrieval

\end{itemize}


\section{Programming skils}

\begin{itemize}

    \item Experienced in Linux/Windows programming based on C/C++, Python, Bash, Emacs, Git.
    \item Experienced in speech processsing toolkit (SPTK, HTK, HTS, Merlin, ESPnet)
    \item Experienced in deep learning framework (PyTorch, Keras)

\end{itemize}


%%%%%%%%%%%%%%%%%%%%%%%%%%%%%%%%%%%%%%%%%%%%%%%%%%%%%%%%%%%%%%%%%%%%%%%%%%%%%%%
\section{Languages}

\TablePad
\begin{tabularx}{\textwidth}{@{}p{0.15\textwidth} p{0.85\textwidth}@{}}
  Japanese & Native
  \\
  English & Intermediate
\end{tabularx}

\section{Memberships}

\begin{itemize}

    \item The Institute of Electrical and Electronics Engineers, Inc. (IEEE), Member
    \item The Acoustical Society of Japan (ASJ), Member

\end{itemize}


%%%%%%%%%%%%%%%%%%%%%%%%%%%%%%%%%%%%%%%%%%%%%%%%%%%%%%%%%%%%%%%%%%%%%%%%%%%%%%%
\section{Awards}

\begin{EntriesTable}
  \Year{2021}  &
  IEEE Signal Processing Society (SPS) Japan Young Author Best Paper Award
  \\
  \Year{2013}  &
  Best Presentation Award in the Acoustic Society of Japan (ASJ)
  \\
  \Year{2012}  &
  Best Presentation Award in the Acoustic Society of Japan (ASJ), Tokai
\end{EntriesTable}


%%%%%%%%%%%%%%%%%%%%%%%%%%%%%%%%%%%%%%%%%%%%%%%%%%%%%%%%%%%%%%%%%%%%%%%%%%%%%%%
% \section{Programming Skills}

% \TablePad
% \begin{tabularx}{\textwidth}{@{}p{0.15\textwidth} p{0.85\textwidth}@{}}
%   Japanese & Native
%   \\
%   English & Intermediate
% \end{tabularx}


%%%%%%%%%%%%%%%%%%%%%%%%%%%%%%%%%%%%%%%%%%%%%%%%%%%%%%%%%%%%%%%%%%%%%%%%%%%%%%%
\section{Publications}

% Book, Journal, Conference

\subsection{Books}


\begin{EntriesTable}
\Year{2021}  &
  \textbf{\MyName}, Shinnosuke Takamichi,
  ``Text-to-speech with Python," Impress.
  \Website{book.impress.co.jp/books/1120101073}
  \GitHub{r9y9/ttslearn}
\end{EntriesTable}

\subsection{Journals}

\begin{EntriesTable}
\Year{2013}  &
  Eita Nakamura, Haruto Takeda, \textbf{\MyName}, Yasuyuki Saito, Shinji Sako, Shigeki Sagayama,
  ``Score Following Handling Performances with Arbitrary Repeats and Skips and Automatic Accompaniment,"
  \emph{Journal of Information Processing Society of Japan}, Vol. 54, No. 4, pp. 1338-1349, 2013.
\end{EntriesTable}


\subsection{Conference proceedings (peer-reviewed)}

\begin{EntriesTable}
  \Year{2022} &
  Byeongseon Park, \textbf{Ryuichi Yamamoto}, Kentaro Tachibana, and Min-Jae Hwang, ``A Unified Accent Estimation Method Based on Multi-Task Learning for Japanese Text-to-Speech," \emph{Proc. Interspeech}, pp. 1931--1935, 2022.
  \Website{6gsn.github.io/demos/mtl\_accent/}
  \\
  ~ &
  Ryo Terashima, \textbf{Ryuichi Yamamoto}, Eunwoo Song, Yuma Shirahata, Hyun-Wook Yoon, Jae-Min Kim, Kentaro Tachibana, ``Cross-Speaker Emotion Transfer for Low-Resource Text-to-Speech Using Non-Parallel Voice Conversion with Pitch-Shift Data Augmentation," \emph{Proc. Interspeech}, pp. 3018--3022, 2022.
  \Website{ryojerky.github.io/demo\_vc-tts-ps/}
  \\
  ~ &
  Eunwoo Song, \textbf{Ryuichi Yamamoto}, Ohsung Kwon, Chan-Ho Song, Min-Jae Hwang, Suhyeon Oh, Hyun-Wook Yoon, Jin-Seob Kim, Jae-Min Kim, ``TTS-by-TTS 2: Data-selective Augmentation for Neural Speech Synthesis Using Ranking Support Vector Machine with Variational Autoencoder," \emph{Proc. Interspeech}, pp. 1941--1945, 2022.
  \Website{sewplay.github.io/demos/txt2/}
  \\
  ~ &
  Takaaki Saeki, Kentaro Tachibana, \textbf{Ryuichi Yamamoto}, ``DRSpeech: Degradation-Robust Text-to-Speech Synthesis with Frame-Level and Utterance-Level Acoustic Representation Learning," \emph{Proc. Interspeech}, pp. 793--797, 2022.
  \Website{takaaki-saeki.github.io/drspeech\_demo/}
  \\
  ~ &
  Hyunwook Yoon, Ohsung Kwon, Hoyeon Lee, \textbf{Ryuichi Yamamoto}, Eunwoo Song, Jae-Min Kim, and Min-Jae Hwang, ``Language Model-Based Emotion Prediction Methods for Emotional Speech Synthesis Systems," \emph{Proc. Interspeech}, pp. 4596--4600, 2022.
  \Website{christophyoon.github.io/lmemotiontts/}
  \\
  \Year{2021} &
  Min-Jae Hwang, \textbf{Ryuichi Yamamoto}, Eunwoo Song, Jae-Min Kim, ``High-Fidelity Parallel WaveGAN with Multi-Band Harmonic-Plus-Noise Model," \emph{Proc. Interspeech}, pp. 2227--2231, 2021.
  \Website{min-jae.github.io/interspeech2021/}
  \\
  ~ &
  Kosuke Futamata, Byeongseon Park, \textbf{Ryuichi Yamamoto}, Kentaro Tachibana, ``Phrase Break Prediction with Bidirectional Encoder Representations in Japanese Text-to-Speech Synthesis," \emph{Proc. Interspeech}, pp. 3126--3130, 2021.
  \Website{matasuke.github.io/demos/pbp\_bert}
  \\
  ~ &
  \textbf{Ryuichi Yamamoto}, Eunwoo Song, Min-Jae Hwang, Jae-Min Kim ``Parallel Waveform Synthesis Based on Generative Adversarial Networks with Voicing-Aware Conditional Discriminators," \emph{Proc. International Conference on Acoustics, Speech and Signal Processing (ICASSP)}, pp. 6039--6043, 2021.
  \Website{r9y9.github.io/demos/projects/icassp2021/}
  \\
  ~ &
  Min-Jae Hwang, \textbf{Ryuichi Yamamoto}, Eunwoo Song, Jae-Min Kim, ``TTS-by-TTS: TTS-Driven Data Augmentation for Fast and High-Quality Speech Synthesis," \emph{Proc. International Conference on Acoustics, Speech and Signal Processing (ICASSP)}, pp. 6598--6602, 2021.
  \Website{min-jae.github.io/icassp2021/}
  \\
  ~ &
  Eunwoo Song, \textbf{Ryuichi Yamamoto}, Min-Jae Hwang, Jin-Seob Kim, Ohsung Kwon, Jae-Min Kim, ``Improved Parallel Wavegan Vocoder with Perceptually Weighted Spectrogram Loss," \emph{Proc. Spoken Language Technology Workshop (SLT)}, pp. 470--476, 2021.
  \Website{sewplay.github.io/demos/wavegan-pwsl/}
  \\
  \Year{2020} &
  Eunwoo Song, Min-Jae Hwang, \textbf{Ryuichi Yamamoto}, Jin-Seob Kim, Ohsung Kwon, Jae-Min Kim, ``Neural Text-to-Speech with a Modeling-by-Generation Excitation Vocoder," \emph{Proc. Interspeech}, pp. 3570--3574, 2020.
  \Website{sewplay.github.io/demos/mbg\_excitnet/}
  \\
  ~ &
  Katsuki Inoue, Sunao Hara, Masanobu Abe, Tomoki Hayashi, \textbf{Ryuichi Yamamoto}, Shinji Watanabe, ``Semi-Supervised Speaker Adaptation for End-to-End Speech Synthesis with Pretrained Models," \emph{Proc. International Conference on Acoustics, Speech and Signal Processing (ICASSP)}, pp. 7634--7638 2020.
  \\
  ~ &
  Min-Jae Hwang, Eunwoo Song, \textbf{Ryuichi Yamamoto}, Frank Soong, Hong-Goo Kang, ``Improving LPCNET-Based Text-to-Speech with Linear Prediction-Structured Mixture Density Network," \emph{Proc. International Conference on Acoustics, Speech and Signal Processing (ICASSP)}, pp. 7219--7223, 2020.
  \\
  ~ &
  \textbf{Ryuichi Yamamoto}, Eunwoo Song, Jae-Min Kim, ``Parallel WaveGAN: A Fast Waveform Generation Model Based on Generative Adversarial Networks with Multi-Resolution Spectrogram," \emph{Proc. International Conference on Acoustics, Speech and Signal Processing (ICASSP)}, pp. 6199--6203, 2020.
  \Website{r9y9.github.io/demos/projects/icassp2020/}
  \\
  ~ &
  Tomoki Hayashi, \textbf{Ryuichi Yamamoto}, Katsuki Inoue, Takenori Yoshimura, Shinji Watanabe, Tomoki Toda, Kazuya Takeda, Yu Zhang, Xu Tan, ``ESPnet-TTS: Unified, Reproducible, and Integratable Open Source End-to-End Text-to-Speech Toolkit," \emph{Proc. International Conference on Acoustics, Speech and Signal Processing (ICASSP)}, pp. 7654--7658, 2020.
  \Website{espnet.github.io/icassp2020-tts/}
  \\
  \Year{2019} &
  \textbf{Ryuichi Yamamoto}, Eunwoo Song, Jae-Min Kim, ``Probability Density Distillation with Generative Adversarial Networks for High-Quality Parallel Waveform Generation," \emph{Proc. Interspeech}, pp. 699--703, 2019.
  \Website{r9y9.github.io/demos/projects/interspeech2019/}
  \\
  ~ &
  Shigeki Karita, Nanxin Chen, Tomoki Hayashi, Takaaki Hori, Hirofumi Inaguma, Ziyan Jiang, Masao Someki, Nelson Enrique Yalta Soplin, \textbf{Ryuichi Yamamoto}, Xiaofei Wang, Shinji Watanabe, Takenori Yoshimura, Wangyou Zhang, ``A Comparative Study on Transformer vs RNN in Speech Applications," \emph{Proc. Automatic Speech Recognition and Understanding Workshop (ASRU)}, pp. 449--456, 2019.
  \\
\Year{2014}  &
  Shinji Sako, \textbf{Ryuichi Yamamoto}, Tadashi Kitamura, ``Ryry: A Real-Time Score-Following Automatic Accompaniment Playback System Capable of Real Performances with Errors, Repeats and Jumps," \emph{Proc. International Conference on Active Media Technology (ICAMT)}, pp. 134--145, 2014.
  \\
\Year{2013}  &
  \textbf{Ryuichi Yamamoto}, Shinji Sako, Tadashi Kitamura, ``Robust On-line Algorithm For Real-time Audio-to-score Alignment Based on A Delayed Decision and Anticipation Framework," \emph{Proc. International Conference on Acoustics, Speech and Signal Processing (ICASSP)}, pp. 191--195, 2013.
  \\
  ~ &
  \textbf{Ryuichi Yamamoto}, Shinji Sako, Tadashi Kitamura, ``Accurate and Low Computational Audio-to-score Alignment Using Segmental CRF with An Explicit Continuous Tempo Model," \emph{Proc. of Communications and Signal Processing (NCSP)}, pp. 345--348, 2013.
\end{EntriesTable}

\subsection{Conference proceedings (non peer-reviewed)}

\begin{EntriesTable}
\Year{2021}  &
  Ryo Terashima, \textbf{Ryuichi Yamamoto}, Kentaro Tachibana, ``An Investigation of Data Augmentation Using CycleGAN Voice Conversion for Text-to-Speech Synthesis," \emph{The Acoustic Society of Japan (ASJ)}, pp. xxx--xxx, 2021 (in Japanese).
  \Website{ryojerky.github.io/demo/}
  \\
\Year{2013}  &
  \textbf{Ryuichi Yamamoto}, Shinji Sako, Tadashi Kitamura, ``Ryry: Automatic Accompaniment System Capable of Polyphonic Instruments," \emph{Proc. Interaction}, 2013 (in Japanese).
  \\
  ~ &
  \textbf{Ryuichi Yamamoto}, Shinji Sako, Tadashi Kitamura, ``Score Following Based on a Combined Model of Score Position and Tempo and Application to Audio-based Automatic Accompaniment," \emph{The Acoustic Society of Japan (ASJ)}, pp. 1065--1066, 2013 (in Japanese).
  \\
\Year{2012}  &
  \textbf{Ryuichi Yamamoto}, Shinji Sako, Tadashi Kitamura, ``Real-time Audio to Score Alignment Using Segmental Conditional Random Fields and Linear Dynamical System," \emph{Proc of The Music Information Retrieval Evaluation eXchange (MIREX)}, 2012.
  \\
  ~ &
  \textbf{Ryuichi Yamamoto}, Shinji Sako, Tadashi Kitamura, ``Audio to Score Alignment Using Semi-Markov Conditional Random Fields," \emph{The Acoustic Society of Japan (ASJ)}, pp. 935--936, 2012 (in Japanese).
  \\
  ~ &
  \textbf{Ryuichi Yamamoto}, Eita Nakamura, Yasuyuki Saito, Shinji Sako, Shigeki Sagayama, ``Eurydice: Automatic Accompaniment System with Jumping Capability," \emph{Proc. Information Processing Society of Japan (IPSJ)}, MUS-96(18), pp. 1--10, 2012 (in Japanese).
  \\
  ~ &
  \textbf{Ryuichi Yamamoto}, Shinji Sako, Tadashi Kitamura, ``Real-time Audio to Score Alignment using Hidden Semi-Markov Model and Linear Dynamical System," \emph{Proc. Information Processing Society of Japan (IPSJ)}, MUS-96(13), pp. 1--6, 2012 (in Japanese).
  \\
  ~ &
  Eita Nakamura, \textbf{Ryuichi Yamamoto}, Shinji Sako, Yasuyuki Saito, Shigeki Sagayama, ``Modeling ornaments in polyphonic MIDI score following and its application to automatic accompaniment", \emph{Proc. The Acoustic Society of Japan (ASJ)}, pp. 929--930, 2012 (in Japanese).
  \\
  ~ &
  Eita Nakamura, \textbf{Ryuichi Yamamoto}, Shinji Sako, Yasuyuki Saito, Shigeki Sagayama, ``Modeling Performance Indeterminacies for Polyphonic Midi Score Following and Its Application to Automatic Accompaniment", \emph{Proc. Information Processing Society of Japan (IPSJ)}, MUS-96(14), pp. 1--6, 2012 (in Japanese).
  \\
\Year{2011}  &
  \textbf{Ryuichi Yamamoto}, Shinji Sako, Tadashi Kitamura, ``Cooperative Automatic Accompaniment System Using Predictive Models of Expression in Music Performance Based on CRFs," \emph{Proc. Information Processing Society of Japan (IPSJ)}, MUS-91(11), pp. 1--6, 2011 (in Japanese).
\end{EntriesTable}

%%%%%%%%%%%%%%%%%%%%%%%%%%%%%%%%%%%%%%%%%%%%%%%%%%%%%%%%%%%%%%%%%%%%%%%%%%%%%%%
\section{Selected Software}

\subsection{Libraries}

\begin{EntriesTable}
  \Duration{2021}{\Ongoing} &
  \textbf{ttslearn}
  \newline
  Library for the book ``Text-to-speech with Python"
  \Role{Creator and core developer}
  \GitHub{r9y9/ttslearn}
  \Website{r9y9.github.io/ttslearn/}
  \\
  \Duration{2020}{\Ongoing} &
  \textbf{nnsvs}
  \newline
  Neural network-based singing voice synthesis library for research
  \Role{Creator and core developer}
  \GitHub{r9y9/nnsvs}
  \\
  \Duration{2017}{\Ongoing} &
  \textbf{nnmnkwii}
  \newline
  Library to build speech synthesis systems designed for easy and fast prototyping
  \Role{Creator and core developer}
  \GitHub{r9y9/nnmnkwii}
  \Website{r9y9.github.io/nnmnkwii/latest/}
  \\
  \Duration{2015}{\Ongoing} &
  \textbf{pysptk}
  \newline
  A python wrapper for Speech Signal Processing Toolkit (SPTK).
  \Role{Creator and core developer}
  \GitHub{r9y9/pysptk}
  \Website{pysptk.readthedocs.io/}
  \\
  \Duration{2015}{\Ongoing} &
  \textbf{pyworld}
  \newline
  A Python wrapper for the high-quality vocoder "World"
  \Role{Core contributor and maintainer}
  \GitHub{JeremyCCHsu/Python-Wrapper-for-World-Vocoder}
  \\
  \Duration{2014}{2020} &
  \textbf{WORLD.jl}
  \newline
  A lightweight Julia wrapper for WORLD - a high-quality speech analysis, modification and synthesis system
  \Role{Creator and core developer}
  \GitHub{r9y9/WORLD.jl}
  \Website{r9y9.github.io/world.jl/latest/}
  \\
  \Duration{2014}{2017} &
  \textbf{MelGeneralizedCepstrums.jl}
  \newline
  Mel-Generalized Cepstrum analysis
  \Role{Creator and core developer}
  \GitHub{r9y9/MelGeneralizedCepstrums.jl}
  \\
  \Duration{2015}{2019} &
  \textbf{librosa}
  \newline
  Python library for audio and music analysis.
  \Role{Contributor}
  \GitHub{librosa/librosa}
  \Website{librosa.org/}
\end{EntriesTable}

\subsection{Research projects}

\begin{EntriesTable}
  \Duration{2019}{\Ongoing} &
  \textbf{ESPnet}
  \newline
  End-to-End Speech Processing Toolkit
  \Role{Discussions and reviews for text-to-speech features}
  \GitHub{espnet/espnet}
  \Website{espnet.github.io/espnet/}
  \\
  \Duration{2017}{2021} &
  \textbf{wavenet\_vocoder}
  \newline
  WaveNet vocoder: neural network based waveform generation models
  \Role{Creator and core developer}
  \GitHub{r9y9/wavenet\_vocoder}
  \Website{r9y9.github.io/wavenet\_vocoder/}
  \\
  \Duration{2017}{2020} &
  \textbf{deepvoice3\_pytorch}
  \newline
  PyTorch implementation of convolutional neural networks-based text-to-speech synthesis models
  \Role{Creator and core developer}
  \GitHub{r9y9/deepvoice3\_pytorch}
  \Website{r9y9.github.io/deepvoice3\_pytorch/}
  \\
  \Duration{2017}{2020} &
  \textbf{gantts}
  \newline
  PyTorch implementation of GAN-based text-to-speech synthesis and voice conversion
  \Role{Creator and core developer}
  \GitHub{r9y9/gantts}
  \\
  \Duration{2017}{2019} &
  \textbf{tacotron\_pytorch}
  \newline
  PyTorch implementation of Tacotron speech synthesis model
  \Role{Creator and core developer}
  \GitHub{r9y9/tacotron\_pytorch}
  \\
  \Duration{2014}{2016} &
  \textbf{VoiceConversion.jl}
  \newline
  Statistical Voice Conversion in Julia
  \Role{Creator and core developer}
  \GitHub{r9y9/VoiceConversion.jl}
  \\
  \Duration{2014}{2016} &
  \textbf{RobustPCA.jl}
  \newline
  Robust Principal Component Analysis in Julia
  \Role{Creator and core developer}
  \GitHub{r9y9/RobustPCA.jl}
\end{EntriesTable}

%%%%%%%%%%%%%%%%%%%%%%%%%%%%%%%%%%%%%%%%%%%%%%%%%%%%%%%%%%%%%%%%%%%%%%%%%%%%%%%
\section{Supervisor for students}

\TablePad
\begin{tabularx}{\textwidth}{@{}p{0.2\textwidth} p{0.85\textwidth}@{}}
  \fontsize{10pt}{0}\selectfont 2021.09 -- present & Reo Yoneyama (Nagoya University)
  \\
  \fontsize{10pt}{0}\selectfont 2021.03 -- 2022.04 & Takaaki Saeki (The University of Tokyo)
\end{tabularx}


%%%%%%%%%%%%%%%%%%%%%%%%%%%%%%%%%%%%%%%%%%%%%%%%%%%%%%%%%%%%%%%%%%%%%%%%%%%%%%%
\section{Presentations}

\begin{EntriesTable}
  \Year{2022} &
  \textbf{Ryuichi Yamamoto}, ``Research and development for TTS in industry,” Nagoya Institute of Technology, Pattern Recognition III, Jan 2022, Online.
  \\
  \Year{2021} &
  Tomohiro Tanaka, \textbf{Ryuichi Yamamoto}, ``Report on Participation in Interspeech2021,” SIG Technical Reports, Dec 2021, Online.
  \\
  \Year{2020} &
  \textbf{Ryuichi Yamamoto}, ``Parallel WaveGAN: Fast and High-Quality GPU Text-to-Speech,” Conference on Computer Science for Enterprise (CCSE), Dec 2020, Online.
  \\
  ~ &
  \textbf{Ryuichi Yamamoto}, ``Parallel WaveGAN: Fast and High-Quality GPU Text-to-Speech,” Main Sesson in LINE DEVELOPER DAY, Nov 2020, Online.
  \Youtube{knzT7M6qsl0}
  \\
  ~ &
  Togami Masahito, Yusuke Kida, \textbf{Ryuichi Yamamoto}, Keisuke Imoto, ``Current progress on speech technologies and its future prospects,” Panel Discussion in LINE DEVELOPER DAY, Nov 2020, Online.
  \Youtube{iSPBCot6n7g}
  \\
  ~ &
  Tomoki Hayashi, \textbf{Ryuichi Yamamoto}, Katsuki Inoue, Takenori Yoshimura, Kazuya Takemura, Tomoki Toda, Shinji Watanabe, ``ESPnet-TTS: A toolkit to accelerate research on end-to-end speech synthesis.,” Special session of The Acoustic Society of Japan (ASJ), Mar 2020, Online.
  \Invited{}
  \Website{kan-bayashi.github.io/asj-espnet2-tutorial/}
  \\
  \Year{2018}  &
  \textbf{Ryuichi Yamamoto}, ``WaveNet: A Generative Model for Raw Audio: What I Learned from Developing An Open-Source Implementation,” Invited Talk in National Institute of Information and Communications Technology (NICT), Feb 2018, Kyoto.
  \Invited{}
  \\
  ~ &
  \textbf{Ryuichi Yamamoto}, ``An Attempt to Reproduce WaveNet-based Text-to-Speech Synthesis,” MACHINE LEARNING Meetup KANSAI, Jun 2018, Kyoto.
  \\
  \Year{2016}  &
  \textbf{Ryuichi Yamamoto}, ``The Julia C++ Interface,” JuliaTokyo \#6, Sep 2016, Tokyo.
  \\
  \Year{2015}  &
  \textbf{Ryuichi Yamamoto}, ``Speech Signal Processing in Julia,” JuliaTokyo \#3, Apr 2015, Tokyo.
  \\
  \Year{2014}  &
  \textbf{Ryuichi Yamamoto}, ``BinDeps.jl,” JuliaTokyo \#2, Sep 2014, Tokyo.
\end{EntriesTable}


\end{document}
